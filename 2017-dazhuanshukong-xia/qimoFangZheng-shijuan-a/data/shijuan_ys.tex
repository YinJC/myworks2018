\usepackage[utf8]{inputenc}

\usepackage[paperwidth=36.8cm,paperheight=26cm,
top=2cm,bottom=2cm,right=2cm,left=3.cm,
columnsep=1.5cm]{geometry}
\columnseprule=.4pt

\usepackage{bbding}
\usepackage{amsmath}
\usepackage{amsfonts}
\usepackage{amssymb}
\usepackage{wasysym}
\usepackage{makeidx}

\usepackage{graphicx}
\usepackage{setspace}
\usepackage{tabu}
\usepackage{paralist}
\usepackage{lastpage}
\usepackage{enumerate} 

%%%% 设置listings宏包用来粘贴源代码
%% 方便粘贴源代码,部分代码高亮功能
\usepackage{listings}

%% 所要粘贴代码的编程语言
%\lstloadlanguages{}

%% 设置listings宏包的一些全局样式
%% 参考http://hi.baidu.com/shawpinlee/blog/item/9ec431cbae28e41cbe09e6e4.html
\lstset{
	showstringspaces=false,              %% 设定是否显示代码之间的空格符号
	numbers=left,                        %% 在左边显示行号
	numberstyle=\tiny,                   %% 设定行号字体的大小
	basicstyle=\footnotesize,                    %% 设定字体大小\tiny, \small, \Large等等
	keywordstyle=\color{blue!70}, commentstyle=\color{red!50!green!50!blue!50},
	%% 关键字高亮
	frame=shadowbox,                     %% 给代码加框
	rulesepcolor=\color{red!20!green!20!blue!20},
	escapechar=`,                        %% 中文逃逸字符,用于中英混排
	xleftmargin=2em,xrightmargin=2em, aboveskip=1em,
	breaklines,                          %% 这条命令可以让LaTeX自动将长的代码行换行排版
	extendedchars=false                  %% 这一条命令可以解决代码跨页时,章节标题,页眉等汉字不显示的问题
}
%%%% listings宏包设置结束


\usepackage{fancyhdr}
\renewcommand{\headrulewidth}{0pt}
\pagestyle{fancy}
\fancyfoot[CO,CE]{第~\thepage~页~~共~\pageref{LastPage}~页}
\fancyhead[RE]{\leavevmode\vbox to0pt{
		\vss\rlap{\putzdxx }\vskip -26cm }} %奇数页眉的右边                       
\fancyhead[LO]{\leavevmode\vbox to0pt{
		\vss\rlap{\putzdx }\vskip -26cm }} %偶数页眉的左边
\renewcommand{\headrulewidth}{0.5pt} 
\fancyhead[C]{《数控编程与操作》课程期未考试试卷 } %页眉中小

\newsavebox{\zdxa}%装订线

\sbox{\zdxa}
{	\parbox{27cm}{\centering \heiti \hspace{1cm}
		系~部:\underline{\makebox[25mm][c]{}}~~~
		专~业:\underline{\makebox[25mm][c]{}}~~~ 班~级:\underline{\makebox[45mm][c]{}}~~~  学~号:\underline{\makebox[25mm][c]{}}~~~
		姓~名:\underline{\makebox[25mm][c]{}} \\
		\vspace{1mm}
%		请在所附答题纸上空出密封位置。并填写试卷序号、班级、学号和 姓名\\
		%答题时学号
%		\vspace{1mm}
	  \dotfill{} 〇 \dotfill{} 密\dotfill{}封\dotfill{}线\dotfill{}〇\dotfill{} \\
}}
\newsavebox{\zdxb}%装订线
\sbox{\zdxb}
{\parbox{27cm}{\centering \heiti
		\vspace{25mm}
		装~~~~~~订~~~~~~区~~~~~~~~~~~~
		装~~~~~~订~~~~~~区~~~~~~ ~~~~~~
		装~~~~~~订~~~~~~区~~~~~~~~~~~~
		装~~~~~~订~~~~~~区\\
		\vspace{6mm}
		\dotfill{} 〇 \dotfill{}密\dotfill{}封\dotfill{}线\dotfill{}〇\dotfill{} \\
}}

\newcommand{\putzdx}{
		\hspace{-1.7cm}\parbox{1cm}{\vspace{-1.5cm}
			\rotatebox[origin=c]{90}{
				\usebox{\zdxa}
		}}
}
\newcommand{\putzdxx}{
	\hspace{0.3cm}\parbox{1cm}{\vspace{-1.5cm}
		\rotatebox[origin=c]{-90}{
			\usebox{\zdxb}
	}}
}


\usepackage{ifthen}

%选择题选项命令 \xx \xxiii \xxv \xxvi
\newlength{\la}
\newlength{\lb}
\newlength{\lc}
\newlength{\ld}
\newlength{\lee}
\newlength{\lf}
\newlength{\lhalf}
\newlength{\lquarter}
\newlength{\lmax}
\newcommand{\xx}[4]{\\[.5pt]%  
	\settowidth{\la}{A、#1~~~}  
	\settowidth{\lb}{B、#2~~~}
	\settowidth{\lc}{C、#3~~~}  
	\settowidth{\ld}{D、#4~~~}  
	\ifthenelse{\lengthtest{\la > \lb}}
	{\setlength{\lmax}{\la}}{\setlength{\lmax}{\lb}}  
		\ifthenelse{\lengthtest{\lmax < \lc}}  {\setlength{\lmax}{\lc}}  {}  \ifthenelse{\lengthtest{\lmax < \ld}}  {\setlength{\lmax}{\ld}}  {} 
	    \setlength{\lhalf}{0.5\linewidth} 
	    \setlength{\lquarter}{0.25\linewidth}
	    \ifthenelse{\lengthtest{\lmax > \lhalf}}  
	    {\noindent{}A、#1 \\ B、#2 \\ C、#3 \\ D、#4 }  {  \ifthenelse{\lengthtest{\lmax > \lquarter}}  
	    	{\noindent
	    	 \makebox[\lhalf][l]{A、#1~~~}% 
	    	 \makebox[\lhalf][l]{B、#2~~~}\\%
	    	 \makebox[\lhalf][l]{C、#3~~~}%
	    	 \makebox[\lhalf][l]{D、#4~~~}}% 
    		 {\noindent\makebox[\lquarter][l]{A、#1~~~}% 
    		 \makebox[\lquarter][l]{B、#2~~~}%     
    		 \makebox[\lquarter][l]{C、#3~~~}%      
    		 \makebox[\lquarter][l]{D、#4~~~}}
    	 }}
     
\newcommand{\xxiii}[3]{\\[.5pt]%  
	\settowidth{\la}{A、#1~~~}  
	\settowidth{\lb}{B、#2~~~}
	\settowidth{\lc}{C、#3~~~}  
	\ifthenelse{\lengthtest{\la > \lb}}
	{\setlength{\lmax}{\la}}{\setlength{\lmax}{\lb}}  
	\ifthenelse{\lengthtest{\lmax < \lc}}  {\setlength{\lmax}{\lc}}  {}  
	\setlength{\lhalf}{0.5\linewidth} 
	\setlength{\lquarter}{0.25\linewidth}
	\ifthenelse{\lengthtest{\lmax > \lhalf}}  
	{\noindent{}A、#1 \\ B、#2 \\ C、#3 }  {  \ifthenelse{\lengthtest{\lmax > \lquarter}}  
		{\noindent
			\makebox[\lhalf][l]{A、#1~~~}% 
			\makebox[\lhalf][l]{B、#2~~~}\\%
			\makebox[\lhalf][l]{C、#3~~~}}% 
		{\noindent
			\makebox[\lquarter][l]{A、#1~~~}% 
			\makebox[\lquarter][l]{B、#2~~~}%     
			\makebox[\lquarter][l]{C、#3~~~}}
}}

\newcommand{\xxv}[5]{\\[.5pt]%  
	\settowidth{\la}{A、#1~~~}  
	\settowidth{\lb}{B、#2~~~}
	\settowidth{\lc}{C、#3~~~} 
	\settowidth{\ld}{D、#4~~~}  
	\settowidth{\lee}{E、#5~~~}   
	\ifthenelse{\lengthtest{\la > \lb}}
	{\setlength{\lmax}{\la}}{\setlength{\lmax}{\lb}}  
	\ifthenelse{\lengthtest{\lmax < \lc}}  {\setlength{\lmax}{\lc}}  {} 
	\ifthenelse{\lengthtest{\lmax < \ld}}  {\setlength{\lmax}{\ld}}  {} 
	\ifthenelse{\lengthtest{\lmax < \lee}}  {\setlength{\lmax}{\lee}}  {} 
	\setlength{\lhalf}{0.5\linewidth} 
	\setlength{\lquarter}{0.25\linewidth}
	\ifthenelse{\lengthtest{\lmax > \lhalf}}  
	{\noindent{}A、#1 \\ B、#2 \\ C、#3 \\ D、#4 \\ E、#5}  {  \ifthenelse{\lengthtest{\lmax > \lquarter}}  
		{\noindent
			\makebox[\lhalf][l]{A、#1~~~}% 
			\makebox[\lhalf][l]{B、#2~~~}\\%
			\makebox[\lhalf][l]{C、#3~~~}%
		    \makebox[\lhalf][l]{D、#4~~~}\\%
		    \makebox[\lhalf][l]{E、#5~~~}}% 
		{\noindent
			\makebox[\lquarter][l]{A、#1~~~}% 
			\makebox[\lquarter][l]{B、#2~~~}%     
			\makebox[\lquarter][l]{C、#3~~~}%
		    \makebox[\lquarter][l]{D、#4~~~}\\%
	        \makebox[\lquarter][l]{E、#5~~~}}
}}

\newcommand{\xxvi}[6]{\\[.5pt]%  
	\settowidth{\la}{A、#1~~~}  
	\settowidth{\lb}{B、#2~~~}
	\settowidth{\lc}{C、#3~~~} 
	\settowidth{\ld}{D、#4~~~}  
	\settowidth{\lee}{E、#5~~~}
	\settowidth{\lf}{E、#6~~~}     
	\ifthenelse{\lengthtest{\la > \lb}}
	{\setlength{\lmax}{\la}}{\setlength{\lmax}{\lb}}  
	\ifthenelse{\lengthtest{\lmax < \lc}}  {\setlength{\lmax}{\lc}}  {} 
	\ifthenelse{\lengthtest{\lmax < \ld}}  {\setlength{\lmax}{\ld}}  {} 
	\ifthenelse{\lengthtest{\lmax < \lee}}  {\setlength{\lmax}{\lee}}  {}
    \ifthenelse{\lengthtest{\lmax < \lf}}  {\setlength{\lmax}{\lf}}  {}  
	\setlength{\lhalf}{0.5\linewidth} 
	\setlength{\lquarter}{0.25\linewidth}
	\ifthenelse{\lengthtest{\lmax > \lhalf}}  
	{\noindent{}A、#1 \\ B、#2 \\ C、#3 \\ D、#4 \\ E、#5 \\ F、#6}  {  \ifthenelse{\lengthtest{\lmax > \lquarter}}  
		{\noindent
			\makebox[\lhalf][l]{A、#1~~~}% 
			\makebox[\lhalf][l]{B、#2~~~}\\%
			\makebox[\lhalf][l]{C、#3~~~}%
			\makebox[\lhalf][l]{D、#4~~~}\\%
			\makebox[\lhalf][l]{E、#5~~~}%
		    \makebox[\lhalf][l]{F、#6~~~}}% 
		{\noindent
			\makebox[\lquarter][l]{A、#1~~~}% 
			\makebox[\lquarter][l]{B、#2~~~}%     
			\makebox[\lquarter][l]{C、#3~~~}%
			\makebox[\lquarter][l]{D、#4~~~}\\%
			\makebox[\lquarter][l]{E、#5~~~}%
		    \makebox[\lquarter][l]{F、#6~~~}}%
}}

%填空题画线  \tk
%\newcommand{\tk}[2][2.5]{\; \underline{\hspace{#1 cm} \hphantom{#2} \hspace{#1 cm} } \, }


%判断题后面加括号
\newcommand{\pd}[2][1]{\nolinebreak\dotfill\mbox{\raisebox{-1.8pt}
		{$\cdots$}(\makebox[#1 cm][c]{
		\ifthenelse{\boolean{print}}
		{\ifthenelse{\equal{#2}{t}}{\Checkmark}{\XSolid}}
		{}
		})}}

\newboolean{print}
\setboolean{print}{true}

\usepackage{ulem}
\newcommand{\tk}[2][0.5]{\;\uline{ 
		\hspace*{#1 cm}
		 \ifthenelse{\boolean{print}}{#2}{\hphantom{#2}}  
	 	 \hspace*{#1 cm}
 	 } }
  
\newcommand{\jd}[2][4]{\par
\begin{minipage}[t][#1cm][t]{0.92\linewidth}	
	\ifthenelse{\boolean{print}}{#2}{}  
\end{minipage} 
}	


