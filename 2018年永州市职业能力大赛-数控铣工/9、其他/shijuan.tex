\documentclass[12pt,twocolumn,landscape,UTF8,twoside]{ctexart}
\usepackage[utf8]{inputenc}

\usepackage[paperwidth=36.8cm,paperheight=26cm,
top=2cm,bottom=2cm,right=2cm,left=3.cm,
columnsep=1.5cm]{geometry}
\columnseprule=.4pt

\usepackage{bbding}
\usepackage{amsmath}
\usepackage{amsfonts}
\usepackage{amssymb}
\usepackage{wasysym}
\usepackage{makeidx}

\usepackage{graphicx}
\usepackage{setspace}
\usepackage{tabu}
\usepackage{paralist}
\usepackage{lastpage}
\usepackage{enumerate} 

%%%% 设置listings宏包用来粘贴源代码
%% 方便粘贴源代码,部分代码高亮功能
\usepackage{listings}

%% 所要粘贴代码的编程语言
%\lstloadlanguages{}

%% 设置listings宏包的一些全局样式
%% 参考http://hi.baidu.com/shawpinlee/blog/item/9ec431cbae28e41cbe09e6e4.html
\lstset{
	showstringspaces=false,              %% 设定是否显示代码之间的空格符号
	numbers=left,                        %% 在左边显示行号
	numberstyle=\tiny,                   %% 设定行号字体的大小
	basicstyle=\footnotesize,                    %% 设定字体大小\tiny, \small, \Large等等
	keywordstyle=\color{blue!70}, commentstyle=\color{red!50!green!50!blue!50},
	%% 关键字高亮
	frame=shadowbox,                     %% 给代码加框
	rulesepcolor=\color{red!20!green!20!blue!20},
	escapechar=`,                        %% 中文逃逸字符,用于中英混排
	xleftmargin=2em,xrightmargin=2em, aboveskip=1em,
	breaklines,                          %% 这条命令可以让LaTeX自动将长的代码行换行排版
	extendedchars=false                  %% 这一条命令可以解决代码跨页时,章节标题,页眉等汉字不显示的问题
}
%%%% listings宏包设置结束


\usepackage{fancyhdr}
\renewcommand{\headrulewidth}{0pt}
\pagestyle{fancy}
\fancyfoot[CO,CE]{第~\thepage~页~~共~\pageref{LastPage}~页}
\fancyhead[RE]{\leavevmode\vbox to0pt{
		\vss\rlap{\putzdxx }\vskip -26cm }} %奇数页眉的右边                       
\fancyhead[LO]{\leavevmode\vbox to0pt{
		\vss\rlap{\putzdx }\vskip -26cm }} %偶数页眉的左边
\renewcommand{\headrulewidth}{0.5pt} 
\fancyhead[C]{《数控编程与操作》课程期未考试试卷 } %页眉中小

\newsavebox{\zdxa}%装订线

\sbox{\zdxa}
{	\parbox{27cm}{\centering \heiti \hspace{1cm}
		系~部:\underline{\makebox[25mm][c]{}}~~~
		专~业:\underline{\makebox[25mm][c]{}}~~~ 班~级:\underline{\makebox[45mm][c]{}}~~~  学~号:\underline{\makebox[25mm][c]{}}~~~
		姓~名:\underline{\makebox[25mm][c]{}} \\
		\vspace{1mm}
%		请在所附答题纸上空出密封位置。并填写试卷序号、班级、学号和 姓名\\
		%答题时学号
%		\vspace{1mm}
	  \dotfill{} 〇 \dotfill{} 密\dotfill{}封\dotfill{}线\dotfill{}〇\dotfill{} \\
}}
\newsavebox{\zdxb}%装订线
\sbox{\zdxb}
{\parbox{27cm}{\centering \heiti
		\vspace{25mm}
		装~~~~~~订~~~~~~区~~~~~~~~~~~~
		装~~~~~~订~~~~~~区~~~~~~ ~~~~~~
		装~~~~~~订~~~~~~区~~~~~~~~~~~~
		装~~~~~~订~~~~~~区\\
		\vspace{6mm}
		\dotfill{} 〇 \dotfill{}密\dotfill{}封\dotfill{}线\dotfill{}〇\dotfill{} \\
}}

\newcommand{\putzdx}{
		\hspace{-1.7cm}\parbox{1cm}{\vspace{-1.5cm}
			\rotatebox[origin=c]{90}{
				\usebox{\zdxa}
		}}
}
\newcommand{\putzdxx}{
	\hspace{0.3cm}\parbox{1cm}{\vspace{-1.5cm}
		\rotatebox[origin=c]{-90}{
			\usebox{\zdxb}
	}}
}


\usepackage{ifthen}

%选择题选项命令 \xx \xxiii \xxv \xxvi
\newlength{\la}
\newlength{\lb}
\newlength{\lc}
\newlength{\ld}
\newlength{\lee}
\newlength{\lf}
\newlength{\lhalf}
\newlength{\lquarter}
\newlength{\lmax}
\newcommand{\xx}[4]{\\[.5pt]%  
	\settowidth{\la}{A、#1~~~}  
	\settowidth{\lb}{B、#2~~~}
	\settowidth{\lc}{C、#3~~~}  
	\settowidth{\ld}{D、#4~~~}  
	\ifthenelse{\lengthtest{\la > \lb}}
	{\setlength{\lmax}{\la}}{\setlength{\lmax}{\lb}}  
		\ifthenelse{\lengthtest{\lmax < \lc}}  {\setlength{\lmax}{\lc}}  {}  \ifthenelse{\lengthtest{\lmax < \ld}}  {\setlength{\lmax}{\ld}}  {} 
	    \setlength{\lhalf}{0.5\linewidth} 
	    \setlength{\lquarter}{0.25\linewidth}
	    \ifthenelse{\lengthtest{\lmax > \lhalf}}  
	    {\noindent{}A、#1 \\ B、#2 \\ C、#3 \\ D、#4 }  {  \ifthenelse{\lengthtest{\lmax > \lquarter}}  
	    	{\noindent
	    	 \makebox[\lhalf][l]{A、#1~~~}% 
	    	 \makebox[\lhalf][l]{B、#2~~~}\\%
	    	 \makebox[\lhalf][l]{C、#3~~~}%
	    	 \makebox[\lhalf][l]{D、#4~~~}}% 
    		 {\noindent\makebox[\lquarter][l]{A、#1~~~}% 
    		 \makebox[\lquarter][l]{B、#2~~~}%     
    		 \makebox[\lquarter][l]{C、#3~~~}%      
    		 \makebox[\lquarter][l]{D、#4~~~}}
    	 }}
     
\newcommand{\xxiii}[3]{\\[.5pt]%  
	\settowidth{\la}{A、#1~~~}  
	\settowidth{\lb}{B、#2~~~}
	\settowidth{\lc}{C、#3~~~}  
	\ifthenelse{\lengthtest{\la > \lb}}
	{\setlength{\lmax}{\la}}{\setlength{\lmax}{\lb}}  
	\ifthenelse{\lengthtest{\lmax < \lc}}  {\setlength{\lmax}{\lc}}  {}  
	\setlength{\lhalf}{0.5\linewidth} 
	\setlength{\lquarter}{0.25\linewidth}
	\ifthenelse{\lengthtest{\lmax > \lhalf}}  
	{\noindent{}A、#1 \\ B、#2 \\ C、#3 }  {  \ifthenelse{\lengthtest{\lmax > \lquarter}}  
		{\noindent
			\makebox[\lhalf][l]{A、#1~~~}% 
			\makebox[\lhalf][l]{B、#2~~~}\\%
			\makebox[\lhalf][l]{C、#3~~~}}% 
		{\noindent
			\makebox[\lquarter][l]{A、#1~~~}% 
			\makebox[\lquarter][l]{B、#2~~~}%     
			\makebox[\lquarter][l]{C、#3~~~}}
}}

\newcommand{\xxv}[5]{\\[.5pt]%  
	\settowidth{\la}{A、#1~~~}  
	\settowidth{\lb}{B、#2~~~}
	\settowidth{\lc}{C、#3~~~} 
	\settowidth{\ld}{D、#4~~~}  
	\settowidth{\lee}{E、#5~~~}   
	\ifthenelse{\lengthtest{\la > \lb}}
	{\setlength{\lmax}{\la}}{\setlength{\lmax}{\lb}}  
	\ifthenelse{\lengthtest{\lmax < \lc}}  {\setlength{\lmax}{\lc}}  {} 
	\ifthenelse{\lengthtest{\lmax < \ld}}  {\setlength{\lmax}{\ld}}  {} 
	\ifthenelse{\lengthtest{\lmax < \lee}}  {\setlength{\lmax}{\lee}}  {} 
	\setlength{\lhalf}{0.5\linewidth} 
	\setlength{\lquarter}{0.25\linewidth}
	\ifthenelse{\lengthtest{\lmax > \lhalf}}  
	{\noindent{}A、#1 \\ B、#2 \\ C、#3 \\ D、#4 \\ E、#5}  {  \ifthenelse{\lengthtest{\lmax > \lquarter}}  
		{\noindent
			\makebox[\lhalf][l]{A、#1~~~}% 
			\makebox[\lhalf][l]{B、#2~~~}\\%
			\makebox[\lhalf][l]{C、#3~~~}%
		    \makebox[\lhalf][l]{D、#4~~~}\\%
		    \makebox[\lhalf][l]{E、#5~~~}}% 
		{\noindent
			\makebox[\lquarter][l]{A、#1~~~}% 
			\makebox[\lquarter][l]{B、#2~~~}%     
			\makebox[\lquarter][l]{C、#3~~~}%
		    \makebox[\lquarter][l]{D、#4~~~}\\%
	        \makebox[\lquarter][l]{E、#5~~~}}
}}

\newcommand{\xxvi}[6]{\\[.5pt]%  
	\settowidth{\la}{A、#1~~~}  
	\settowidth{\lb}{B、#2~~~}
	\settowidth{\lc}{C、#3~~~} 
	\settowidth{\ld}{D、#4~~~}  
	\settowidth{\lee}{E、#5~~~}
	\settowidth{\lf}{E、#6~~~}     
	\ifthenelse{\lengthtest{\la > \lb}}
	{\setlength{\lmax}{\la}}{\setlength{\lmax}{\lb}}  
	\ifthenelse{\lengthtest{\lmax < \lc}}  {\setlength{\lmax}{\lc}}  {} 
	\ifthenelse{\lengthtest{\lmax < \ld}}  {\setlength{\lmax}{\ld}}  {} 
	\ifthenelse{\lengthtest{\lmax < \lee}}  {\setlength{\lmax}{\lee}}  {}
    \ifthenelse{\lengthtest{\lmax < \lf}}  {\setlength{\lmax}{\lf}}  {}  
	\setlength{\lhalf}{0.5\linewidth} 
	\setlength{\lquarter}{0.25\linewidth}
	\ifthenelse{\lengthtest{\lmax > \lhalf}}  
	{\noindent{}A、#1 \\ B、#2 \\ C、#3 \\ D、#4 \\ E、#5 \\ F、#6}  {  \ifthenelse{\lengthtest{\lmax > \lquarter}}  
		{\noindent
			\makebox[\lhalf][l]{A、#1~~~}% 
			\makebox[\lhalf][l]{B、#2~~~}\\%
			\makebox[\lhalf][l]{C、#3~~~}%
			\makebox[\lhalf][l]{D、#4~~~}\\%
			\makebox[\lhalf][l]{E、#5~~~}%
		    \makebox[\lhalf][l]{F、#6~~~}}% 
		{\noindent
			\makebox[\lquarter][l]{A、#1~~~}% 
			\makebox[\lquarter][l]{B、#2~~~}%     
			\makebox[\lquarter][l]{C、#3~~~}%
			\makebox[\lquarter][l]{D、#4~~~}\\%
			\makebox[\lquarter][l]{E、#5~~~}%
		    \makebox[\lquarter][l]{F、#6~~~}}%
}}

%填空题画线  \tk
%\newcommand{\tk}[2][2.5]{\; \underline{\hspace{#1 cm} \hphantom{#2} \hspace{#1 cm} } \, }


%判断题后面加括号
\newcommand{\pd}[2][1]{\nolinebreak\dotfill\mbox{\raisebox{-1.8pt}
		{$\cdots$}(\makebox[#1 cm][c]{
		\ifthenelse{\boolean{print}}
		{\ifthenelse{\equal{#2}{t}}{\Checkmark}{\XSolid}}
		{}
		})}}

\newboolean{print}
\setboolean{print}{true}

\usepackage{ulem}
\newcommand{\tk}[2][0.5]{\;\uline{ 
		\hspace*{#1 cm}
		 \ifthenelse{\boolean{print}}{#2}{\hphantom{#2}}  
	 	 \hspace*{#1 cm}
 	 } }
  
\newcommand{\jd}[2][4]{\par
\begin{minipage}[t][#1cm][t]{0.92\linewidth}	
	\ifthenelse{\boolean{print}}{#2}{}  
\end{minipage} 
}	




%\setboolean{print}{true}
\setboolean{print}{false} %是否打印答案

\author{高星}

\begin{document}
\noindent	
	
\begin{spacing}{2}
		\begin{center}
			\zihao{3} \heiti 
				\bf{2018年永州市职业技能大赛}
											
				\zihao{4}	\underline{~《 数控铣工》~}\,理论竞赛试卷
				
	时间:\uline{\hspace*{0.8 cm}90分钟\hspace*{0.8cm}} ~~~~成绩:\tk[1.5]{}  
\end{center}
\end{spacing}
\vspace{-6mm}
\begin{spacing}{1.089}
	\begin{enumerate} [1、]
%
%
\item[\heiti 一、] {\heiti 选择题(每题1分,共80分)}
%
%	
\item YG8硬质合金其代号后面一般的数字代表\tk{D}的百分比含量。
\xx{碳化钨}{碳化钛}{碳化泥}{钴} 
%
\item 通过主切削刃上某一点,并与该点的切削速度方向垂直的平面称为\tk[0.3]{A}。
\xx{基面}{切削平面}{主剖面}{横向剖面}
%
\item 外径千分尺是测量精度等级\tk{D}的工件尺寸。
\xx{不高于IT10}{IT10-IT11}{不高于IT17}{IT7-IT9}
%
\item 地址编码A的意义是\tk{A}。
\xx{围绕X轴回转运动角度尺寸}{围绕Y轴回转运动角度尺寸}{平行于X轴的第二角度尺寸}{平行于X轴的第二角度尺寸}
%
\item \tk{B}的主要作用是减少后刀面与切削表面之间的摩擦。
\xx{前角}{后角}{螺旋角}{刃倾角}
%
\item 根据加工要求,有些工件并不需要限制其6个自由度, 这种定位方式称为\tk[0.6]{B}。
\xx{欠定位}{不完全定位}{过定位}{完全定位}
%
\item 工艺基准按其功用的不同,可分为定位基准、测量基准和\tk{C}基准三种。
\xx{粗}{精}{装配}{设计}
%
\item 高温合金导热性差,高温强度大,切削时容易粘刀,故铣削高温合金时,后角要大些,前角应取\tk{A}。
\xx{正值}{负值}{ 0°}{均可}
%
\item 闭环控制的数控铣床伺服系统常用大惯量\tk{A}电机作驱动元件。
\xx{直流}{交流}{稳压}{步进}
%
\item 孔的轴线的直线度属于孔的\tk{B}。
\xx{尺寸精度}{形状精度}{位置精度}{表面粗糙度}
%
\item 形成\tk{A}切屑过程比较平稳,切削力波动较小,已加工表面粗糙度较高。
\xx{带状}{节状}{粒状}{崩碎}
%
\item 在逆铣时,工件所受的\tk{D}铣削力的方向始终与进给方向相反。
\xx{切向}{径向}{轴向}{纵向}
%
\item 在工艺过程中安排时效工序的目的,主要是\tk{C}。
\xx{增加刚性}{提高硬度}{消除内应力}{增加强度}
%
\item 液压系统的功率大小与系统的\tk{A}大小有关。
\xx{压力和流量}{压强和面积}{压力和体积}{负载和直径}
%
\item 采用手动夹紧装置时,夹紧机构必须具有\tk{B}性。
\xx{导向}{自锁}{平衡}{平稳}
%
\item 数控铣床作空运转试验的目的是\tk{C}。
\xx{检验加工精度}{检验工率}{检验是否能正常运转}{前几项均不正确}
%
\item 可能有间隙或可能有过盈的配合称为\tk{B}。
\xx{ 间隙}{ 过渡 }{ 过盈}{前几项均不正确}
%
\item 在GB3052-82中规定,机床的某一部件运动的\tk{A}方向,是增大工件和刀具之间距离的方向。
\xx{正方向}{负方向}{各厂家自行制定}{前几项均不正确}
%
\item 切削参数中\tk{C}对切削瘤影响最大。
\xx{进给度}{切削深度}{切削速度}{前几项均不正确}
%
\item 在公制格式下某一段程序为 N50 ~G01~ X120.~F200  说明:\tk{C}。
\xx{执行该程序段  机床X轴移动120mm}{ 执行该程序段 机床X 轴将到达X 120位置上}{ 单独一段,不能说明问题 }{前几项均不正确} 
%
\item 零件图尺寸标注的基准一定是\tk{B}。
\xx{定位基准}{设计基准}{测量基准}{工序基准}
%
\item 主轴回转中心线对工作台面的平行度,若超过公差,则在作纵向进给铣削时,会影响\tk{A}。
\xx{加工面的平行度	}{铣刀耐用度}{加工面的表面粗糙度}{无影响}
%
\item 柔性制造系统简称\tk{B}。
\xx{CAD    }{FMS    }{NC }{ CAM}
%
\item 数控铣床中,滚珠丝杠螺母副是一种新的传动机构,它精密而又\tk{B}故其用途越来越广。
\xx{能自锁}{工艺简单}{省力}{材料}
%
\item 数控铣床中把脉中信号转换成机床移动部件运动的组成部分称为\tk{C}。
\xx{控制介质}{数控装置}{伺服系统}{机床本体}
%
\item 成组零件的工艺路线是按\tk{A}的主样件拟定的。
\xx{零件族}{零件组}{零件}{零件系}
%
\item 下面指令中属于非模态指令的是\tk{C}。
\xx{G90}{G2	}	{G4}	{G99}
%
\item 圆弧插补指令 G17 G3 X\_\_ Y\_\_ R\_\_ F\_\_中的XY表示圆弧的\tk[0.4]{B}。
\xx{起点坐标}	{终点坐标}	{圆心坐标}	{圆心相对于起点的值}
%
\item G00指令与下列的\tk{D}指令不是同一组的。
\xx{G1}		{G2}	{G3}	{G4}
%
\item 确定数控机床的坐标轴时,一般应先确定\tk{C}。
\xx{X轴}	{Y轴} {Z轴}		{U轴}
%
\item 辅助功能中与主轴有关的M指令为\tk{A}。
\xx{M5} {M6}{M9}{M7}
%
\item 若铣削速度为75m/min,铣刀直径为80mm,则铣刀的转速为\tk{B}r/min。
\xx{258}{298}	{358}	{398}
%
\item Fanuc加工中心系统中,用于深孔加工的指令是\tk{A}		。
\xx{G73}{G81}{G82}{G85}
%
\item Fanuc上子程序结束的指令为\tk{C}。
\xx{G99}	{G98}	{M99}	{M98}
%
\item 在Fanuc系统中,在主程序中调用子程序O1000,其正确的指令是\tk{C}		。
\xx{M98 O1000}		{M99 O1000}		{M98 P1000}		{G98 P1000}
%
\item 若要使刀具中心靠近编程轮廓,则刀补的绝对值\tk{B}。
\xxiii{增大}	{减少}	{不变}	
%
\item 用6.2的刀补加工$\diameter 100^{+0.04}_{~\; 0}$的外圆,经测量其值为 \diameter 100.46,侧精加工刀补为\tk{C}	。
\xx{6.0}	{6.43}	{5.98}	{5.97}
%
\item 数控机床的位置精度主要指标有\tk{A}。
\xx{定位精度和重复定位精度}{分辨率和脉冲当量}{主轴回转精度}{几何精度}
%
\item 数控系统中PMC控制程序实现机床的\tk{B}。
\xx{位置控制}{各执行机构的逻辑顺序控制}{插补控制}{各进给轴轨迹和速度控制}
%
\item 对于非圆曲线加工,一般用直线和圆弧逼近,在计算节点时,要保证非圆曲线和逼近直线或圆弧之间的法向距离小于允许的程序编制误差,允许的程序编制误差一般取零件公差的\tk{C}。
\xx{1/2~1/3}{1/3~1/5}{1/5~1/10}{等同值}
%
\item \tk{B}是一种旋转式测量元件,通常装在被检测轴上,随被检测轴一起转动。可将被测轴的角位移转换成增量脉冲形式或绝对式的代码形式。
\xx{旋转变压器}{编码器}{圆光栅}{测速发电机}
%
\item 孔和轴各有\tk{D}个基本偏差。
\xx{20}{28}{18}{26}
%
\item 操作人员确认符合开机的条件后可启动机床,并且使机床空运转\tk{A}左右。
\xx{15分钟}{60分钟}{2小时}{3小时}
%
\item 选择对刀点时应选在零件的\tk{A}。
\xx{设计基准上   }{零件边缘上}{任意位置}{中心位置}
%
\item G00速度是由 \tk{A}决定的。
\xx{机床内参数决定}{操作者输入}{编程}{进给速度}
%
\item 执行直线插补指令G01与\tk{B}无关。
\xx{进给率}{坐标平面的选择}{起点坐标}{机床位置}
%
\item 工作台定位精度测量时应使用\tk{A}。
\xx{激光干涉仪}{百分表}{千分尺}{游标卡尺}
%
\item 由机床的档块和行程开关决定的坐标位置称为\tk{A}。
\xx{机床参考点}{机床原点}{机床换刀点}{刀架参考点}
%
\item  G02 X20 Y20 R-10 F100;所加工的一般是\tk{C}。
\xx{整圆}{夹角〈=180°的圆弧}{180°〈夹角〈360°的圆弧}{夹角〈=90°的圆弧}
%
\item 数控机床四轴三联动的含义是\tk{C}。
\xx{四轴中只有三个轴可以运动 }{有四个控制轴、其中任意三个轴可以联动}{数控系统能控制机床四轴运动,其中三个轴能联动}{前几项均不正确}
%
\item 逐点比较插补法的插补流程是\tk{C}。
\xx{偏差计算→偏差判别→进给→终点判别}{终点判别→进给→偏差计算→偏差判别}{偏差判别→进给→偏差计算→终点判别}{终点判别→进给→偏差判别→偏差计算}
%
\item MBD是指\tk{C}。
\xx{柔性制造系统}{计算机集成制造系统}{基于模型的定义}{自适应控制}
%
\item 加工中心上一般采用\tk{D}圆锥刀柄,这类刀柄不能自锁,换刀方便,且有较高的定心精度和刚度。
\xx{1:16}{7:64}{1:4}{7:24}
%
\item 机械效率值永远是\tk{B}。
\xx{大于1}{小于1}{等于1}{负数}
%
\item 加工精度为IT7级的孔,当孔径小于12mm时可以采用\tk{C}方案。
\xx{钻——扩}{钻——镗}{钻——粗铰——精铰}{钻——扩——铰}
%
\item 为了保障人身安全,在正常情况下,电气设备的安全电压规定为\tk{B}。
\xx{42V}{36V}{24V}{12V}
%
\item 数控机床的核心是\tk{B}。
\xx{伺服系统}{数控系统}{反馈系统}{传动系统}
%
\item 数控加工中心的固定循环功能适用于\tk{C}。
\xx{曲面形状加工}{平面形状加工}{孔系加工}{螺纹加工}
%
\item 在运算指令中,形式为\#i=ROUND[\#j]代表的意义是\tk{B}。
\xx{圆周率}{四舍五入整数化}{圆弧度}{加权平均}
%
\item 在变量赋值方法I中,引数(自变量)A对应的变量是\tk{D}。
\xx{\#101 }{\#31}{\#21}{\#1}
%
\item 在运算指令中,形式为\#i=LN[\#j]代表的意义是\tk{B}。
\xx{长度}{自然对数}{轴距}{位移偏差度}
%
\item 封闭环的最小极限尺寸等于各增环的最小极限尺寸\tk{C}各减环的最大极限尺寸之和。
\xx{之差乘以}{之差除以}{之和减去}{除以}
%
\item 分析零件图的视图时,根据视图布局,首先找出\tk{A}。
\xx{主视图}{后视图}{俯视图}{前视图}
%
\item 变量包括有局部变量、\tk{C}。
\xx{局部变量}{大变量}{公用变量和系统变量}{小变量}
%
\item 控制指令IF[<条件表达式>]GOTO n表示若条件不成立,则转向\tk[0.4]{A}。
\xx{下一句}{n}{n-1}{n+1}
%
\item 用行(层)切法加工空间立体曲面,即三坐标运动、二坐标联动的编程方法称为\tk{C}加工。
\xx{4.5维            }{5.5维}{2.5维}{3.5维}
%
\item 在圆弧逼近零件轮廓的计算中,整个曲线是一系列彼此\tk{B}的圆弧逼近实现的。
\xx{重合}{相交}{相切}{包含}
%
\item 端面多齿盘齿数为72,则分度最小单位为\tk{D}度。
\xx{72}{64}{55}{5}
%
\item 加工中心按照主轴在加工时的空间位置分类,可分为立式、卧式、\tk{D}加工中心。
\xx{不可换主轴箱}{三轴、五面}{复合、四轴}{万能}
%
\item 箱体类零件一般是指\tk{C}孔系,内部有一定型腔,在长、宽、高方向有一定比例的零件。
\xx{至多具有2}{至少具有5个}{具有1个以上}{至少具有11个}
%
\item 已知直线经过$(x_1,y_1)$点,斜率为$k(k≠0)$,则直线方程为\tk{A}。
\xx{$y-y_1=k(x-x_1)$}{$y=kx+8$}{$y=4kx$}{$y=ax+b$}
%
\item 宏程序\tk{C}。
\xx{计算错误率高}{计算功能差,不可用于复杂零件}{可用于加工不规则形状零件}{无逻辑功能}
%
\item 在零件毛坯加工余量不匀或\tk{A}的情况下进行加工,会引起切削力大小的变化,因而产生误差。
\xx{材料硬度变化}{材料硬度无变化}{加工余量非常均匀}{加工余量无变化}
%
\item 一般情况下,直径\tk{A}的孔应由普通机床先粗加工,给加工中心预留余量为4至6mm(直径方向),再由加工中心加工。
\xx{大于$\phi$30mm}{小于$\phi$30mm}{小于$\phi$15mm}{为$\phi$19mm}
%
\item 进行基准重合时的工序尺寸计算,应从\tk{B}道工序算起。
\xx{任意}{最后一}{最开始的第一}{中间的第一}
%
\item  钛的熔点为\tk{C}摄氏度。
\xx{540}{609}{1668 }{550}
%
\item 机床通电后应首先检查\tk{B}是否正常。
\xx{加工路线}{各开关按钮和键}{电压、油压、加工路线}{工件精度}
%
\item 在机械加工时,机床、夹具、刀具和工件构成了一个完整的系统称为\tk[0.3]{C}。
\xx{计算系统}{设计系统}{工艺系统}{测量系统}
%
\item 毛坯的形状误差对下一工序的影响表现为\tk{A}复映。
\xx{误差}{公差}{形状}{形位和状态}
%
\item 成组夹具是适应\tk{D}需要发起来的。
\xx{一般工艺}{车床工艺}{钻床工艺}{成组工艺}

% \newpage

\item[\heiti 二、] { \heiti 判断题(每题1分,共20分,对的打~{\Checkmark}~,错的打~{\XSolid}~)}

\item 高速钢与硬质合金钢相比,具有硬度强、红硬性和耐磨性较好等优点。 \pd{f}
\item YG类硬质合金中含钴量较高的牌号耐磨性较好,硬度较高。\pd{f}
\item 铣床主轴的转速越高,则铣削速度必定越大。\pd{f}
\item 在装夹工件时,为了不使工件产生位移,夹(或压)紧力应尽量大, 越大越好越牢。 \pd{f}
\item 在一定切削速度范围内,切削速度与切削寿命之间在双对数坐标系下是线形关系。 \pd{t}
\item 退火的主要目的是调整钢件的硬度等。\pd{f}
\item 在高温下,刀具切削部分必须具有足够的硬度, 这种在高温下仍具有硬度的性质称为红硬性。 \pd{t}
\item 硬质合金是金属碳化物和以钴为主的金属粘结剂经粉末冶金工艺制造而成的。  \pd{t}
\item 点位控制系统不仅要控制从一点到另一点的准确定位,还要控制从一点到另一点的路径。\pd{f}
\item 槽铣刀的用途是铣削各种槽。\pd{f}
\item 前刀面与主后刀面的交线是副切削刃,担负着主要切削和排屑工作。\pd{f}
\item 检验铣床工作精度,往往用试切试件法,试件的材料是黄铜。\pd{f}
\item 用杠杆卡规可以测量出工件的圆柱度和平行度。\pd{t}
\item 在确定工件在夹具中的定位方案时,决不允许发生欠定位。\pd{t}
\item 杠杆卡规的刻度值根据测量范围分为0.002mm和0.005mm两种。 \pd{t}
\item 杠杆卡规是利用杠杆齿轮放大原理制造的量仪。\pd{f}
\item 分度值为0.02mm/m的水平仪,当气泡偏移零位两格时, 表示被测物体在1m内的长度上高度差为0.02mm。\pd{f}
\item 当工件以一面两销定位时,其中削边销的横截面长轴应平行于两销的中心连线。\pd{f}
\item 一旦某个零件的工艺规程订好以后,必须严格遵照执行,不能任意改变。 \pd{t}
\item 一个主程序调用另一个主程序称为主程序嵌套。\pd{f}

	\end{enumerate} 
	\end{spacing}
\end{document}